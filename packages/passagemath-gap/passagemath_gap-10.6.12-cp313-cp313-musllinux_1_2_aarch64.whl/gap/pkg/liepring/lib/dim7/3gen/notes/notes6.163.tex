
\documentclass[12pt]{article}
%%%%%%%%%%%%%%%%%%%%%%%%%%%%%%%%%%%%%%%%%%%%%%%%%%%%%%%%%%%%%%%%%%%%%%%%%%%%%%%%%%%%%%%%%%%%%%%%%%%%%%%%%%%%%%%%%%%%%%%%%%%%%%%%%%%%%%%%%%%%%%%%%%%%%%%%%%%%%%%%%%%%%%%%%%%%%%%%%%%%%%%%%%%%%%%%%%%%%%%%%%%%%%%%%%%%%%%%%%%%%%%%%%%%%%%%%%%%%%%%%%%%%%%%%%%%
\usepackage{amsfonts}
\usepackage{amssymb}
\usepackage{sw20elba}

%TCIDATA{OutputFilter=LATEX.DLL}
%TCIDATA{Version=5.50.0.2890}
%TCIDATA{<META NAME="SaveForMode" CONTENT="1">}
%TCIDATA{BibliographyScheme=Manual}
%TCIDATA{Created=Saturday, July 20, 2013 14:50:26}
%TCIDATA{LastRevised=Saturday, August 10, 2013 17:47:15}
%TCIDATA{<META NAME="GraphicsSave" CONTENT="32">}
%TCIDATA{<META NAME="DocumentShell" CONTENT="Articles\SW\mrvl">}
%TCIDATA{CSTFile=LaTeX article (bright).cst}

\newtheorem{theorem}{Theorem}
\newtheorem{axiom}[theorem]{Axiom}
\newtheorem{claim}[theorem]{Claim}
\newtheorem{conjecture}[theorem]{Conjecture}
\newtheorem{corollary}[theorem]{Corollary}
\newtheorem{definition}[theorem]{Definition}
\newtheorem{example}[theorem]{Example}
\newtheorem{exercise}[theorem]{Exercise}
\newtheorem{lemma}[theorem]{Lemma}
\newtheorem{notation}[theorem]{Notation}
\newtheorem{problem}[theorem]{Problem}
\newtheorem{proposition}[theorem]{Proposition}
\newtheorem{remark}[theorem]{Remark}
\newtheorem{solution}[theorem]{Solution}
\newtheorem{summary}[theorem]{Summary}
\newenvironment{proof}[1][Proof]{\noindent\textbf{#1.} }{{\hfill $\Box$ \\}}
\input{tcilatex}
\addtolength{\textheight}{30pt}

\begin{document}

\title{Algebras 6.163 - 6.167}
\author{Michael Vaughan-Lee}
\date{July 2013}
\maketitle

Algebras 6.163 -- 6.167 give a classification of algebras of order $p^{6}$
with presentations 
\[
\langle a,b,c\,|\,ca-baa,\,cb,\,pa-\lambda baa-\mu bab,\,pb+\nu baa+\xi
bab,\,pc,\,\text{class }3\rangle 
\]%
with $\lambda ,\mu ,\nu ,\xi \neq 0$. Most of these algebras are terminal,
and we need a slightly different classification of these algebras from that
given in the classification of nilpotent Lie rings of order $p^{6}$, so as
to classify the capable ones. It turns out that $\frac{5}{2}p-\frac{9}{2}+%
\frac{1}{2}\gcd (p-1,4)$ of these algebras are capable, and that they have a
total of $\frac{1}{2}p^{3}+2p^{2}-5p+\frac{1}{2}+\frac{p}{2}\gcd (p-1,4)$
descendants of order $p^{7}$ and $p$-class 4.

Let $L$ have the presentation above, and suppose that $a^{\prime },b^{\prime
},c^{\prime }$ generate $L$ and satisfy similar relations, but with
(possibly) different $\lambda ,\mu ,\nu ,\xi $. Then 
\begin{eqnarray*}
a^{\prime } &=&\alpha a+\gamma c, \\
b^{\prime } &=&\delta b+\varepsilon c, \\
c^{\prime } &=&\alpha \delta c
\end{eqnarray*}%
modulo $L_{2}$ and 
\begin{eqnarray*}
pa^{\prime } &=&\frac{\lambda }{\alpha \delta }b^{\prime }a^{\prime
}a^{\prime }+\frac{\mu }{\delta ^{2}}b^{\prime }a^{\prime }b^{\prime }, \\
pb^{\prime } &=&\frac{\nu }{\alpha ^{2}}b^{\prime }a^{\prime }a^{\prime }+%
\frac{\xi }{\alpha \delta }b^{\prime }a^{\prime }b^{\prime }
\end{eqnarray*}%
or 
\begin{eqnarray*}
a^{\prime } &=&\alpha b+\gamma c, \\
b^{\prime } &=&\delta a+\varepsilon c, \\
c^{\prime } &=&\alpha \delta c
\end{eqnarray*}%
modulo $L_{2}$ and 
\begin{eqnarray*}
pa^{\prime } &=&\frac{\xi }{\alpha \delta }b^{\prime }a^{\prime }a^{\prime }+%
\frac{\nu }{\delta ^{2}}b^{\prime }a^{\prime }b^{\prime }, \\
pb^{\prime } &=&\frac{\mu }{\alpha ^{2}}b^{\prime }a^{\prime }a^{\prime }+%
\frac{\lambda }{\alpha \delta }b^{\prime }a^{\prime }b^{\prime }.
\end{eqnarray*}

So we can take $\lambda =1$ and $\mu =1$ or $\omega $ (or any other fixed
integer which is not a square $\func{mod}p)$. Given these values of $\lambda
,\mu $ it turns out that the algebra is terminal unless $\xi =1$ or $\xi
=\mu \nu $.

So we have two families of capable algebras of order $p^{6}$:

\[
\langle a,b,c\,|\,ca-baa,\,cb,\,pa-baa-\mu bab,\,pb+\nu baa+bab,\,pc,\,\text{%
class }3\rangle , 
\]

\[
\langle a,b,c\,|\,ca-baa,\,cb,\,pa-baa-\mu bab,\,pb+\nu baa+\mu \nu
bab,\,pc,\,\text{class }3\rangle . 
\]

These two families have immediate descendants of order $p^{7}$ with the
following presentations involving parameters $y,z,t$:

\[
\langle a,b,c\,|\,ca-baa,\,cb,\,pa-baa-\mu bab-ybaaa,\,pb+\nu
baa+bab-zbaaa,\,pc-tbaaa,\,\text{class }3\rangle , 
\]

\[
\langle a,b,c\,|\,ca-baa,\,cb,\,pa-baa-\mu bab-ybaaa,\,pb+\nu baa+\mu \nu
bab-zbaaa,\,pc-tbaaa,\,\text{class }3\rangle . 
\]

For the first family of descendants we consider transformations of the form%
\begin{eqnarray*}
a^{\prime } &=&\pm a+\gamma c, \\
b^{\prime } &=&\pm b+\varepsilon c, \\
c^{\prime } &=&c,
\end{eqnarray*}%
where if $\mu \nu =1$ we need $\gamma =\mu \varepsilon $, and
transformations of the form 
\begin{eqnarray*}
a^{\prime } &=&\alpha b+\gamma c, \\
b^{\prime } &=&\alpha ^{-1}a+\varepsilon c, \\
c^{\prime } &=&c,
\end{eqnarray*}%
where $\alpha ^{2}\nu =\mu $, and where if $\mu \nu =1$ we need $\gamma =\mu
\varepsilon $. For these transformations we have%
\begin{eqnarray*}
y &\rightarrow &\pm y+\gamma t+\gamma \mu ^{-1}+\varepsilon  \\
z &\rightarrow &\pm z+\varepsilon t-\nu \gamma \mu ^{-1}+\nu \varepsilon
-2\varepsilon \mu ^{-1}, \\
t &\rightarrow &t,
\end{eqnarray*}%
and%
\begin{eqnarray*}
y &\rightarrow &-\alpha z-\gamma t+\varepsilon -\nu \gamma +2\gamma \mu
^{-1}, \\
z &\rightarrow &-\alpha ^{-1}y-\varepsilon t-\gamma \mu ^{-1}\nu
-\varepsilon \mu ^{-1}, \\
t &\rightarrow &-t-\nu +\mu ^{-1}.
\end{eqnarray*}

For the second family of descendants we can assume that $\mu \nu \neq 1$. We
consider transformations of the form%
\begin{eqnarray*}
a^{\prime } &=&\pm a+\mu \varepsilon c, \\
b^{\prime } &=&\pm b+\varepsilon c, \\
c^{\prime } &=&c,
\end{eqnarray*}%
and, when $\mu \nu =-1$ and $p=1\func{mod}4$, transformations of the form%
\begin{eqnarray*}
a^{\prime } &=&\alpha b+\mu \varepsilon c, \\
b^{\prime } &=&-\alpha ^{-1}a+\varepsilon c, \\
c^{\prime } &=&-c,
\end{eqnarray*}%
where $\alpha ^{2}=-\mu ^{2}$. For these transformations we have%
\begin{eqnarray*}
y &\rightarrow &\pm y+\mu \varepsilon t+2\varepsilon , \\
z &\rightarrow &\pm z+\varepsilon t-2\nu \varepsilon , \\
t &\rightarrow &t,
\end{eqnarray*}%
and%
\begin{eqnarray*}
y &\rightarrow &-\alpha z-\mu \varepsilon t-2\varepsilon , \\
z &\rightarrow &\alpha ^{-1}y-\varepsilon t+2\nu \varepsilon , \\
t &\rightarrow &t.
\end{eqnarray*}

There is a \textsc{Magma} program to compute representative sets of
parameters in notes6.163.m.

\end{document}
